\section{Heatmap}
One of the most easily created and useful spatial data visualizations is a \emph{heatmap}. In our DotA2 dataset we have several matches worth of player coordinates over time. Even more interesting, we have matches from several tier brackets. Once we have a heatmap of all data, it would also be interesting to create a heatmap per tier, to compare the overall map usage per tier. After that several filters were applied to these heatmaps.

When we tried to parse the dataset in the D3 library we encountered a severe problem however: this library is not suited for datasets of this scale. To solve this issue we chose to preprocess our dataset, transforming it into a heatmap dataset usable in D3 using a different platform first. Our second step was to filter only a certain tier in this transformation. The third step was applying several visual filters to these heatmaps.

\subsection{Preprocessing}
We used MATLAB in order to get the heatmap data for the entire data set as well as the individual heatmaps per tier.

The script that is used to extract the data that is required for this processing can be accessed from index.html by clicking \textit{MATLAB extract} This was soleley used to extract the x, y coordinates as well as the corresponding tier, as this is the only required data that is needed to generate the set of heatmaps that we are doing.

Scripts \textit{MATLAB 1} and \textit{MATLAB 2} that can also be viewed from index.html are used to generate the data that is later used to be displayed using D3js.

Note that the scripts are not fully automated and hence need to be reconfigured to obtain the necessary information. For the purpose of this assignment this was unnecessary, however, the entire process can be written in such way so that the extraction and all the necessary pre-processings are done in one easy step.

\subsection{Filtering}
We further experimented with the heatmaps that were generated in the above step by applying several image processing algorithms such as a 4x4 Median Filter and a Canny Edge Algorithm in Matlab which gave very interesting results.

The \textit{4x4 Median Filtered} gives a smoothed out heatmap where the 'heat' regions are in blocks of areas.

On the other hand the \textit{Canny Edge Algorithm} detects the edges in the image, and by inspecting the image we can see that the result is sort of a road map of the game.

In order to inspect the visualizations please navigate to \textit{index.html}

\subsection{Rendering the heatmap in D3}
Since our heatmap dataset consisted of a csv with for every point a triple of the following format: $(x, y, v)$ where x and y encoded the location of this point, and v the count of this position being entered in the dataset by a player, loading the heatmap in D3 was trivial.

\begin{lstlisting}[language=JavaScript]
// Maps raw map coordinates into normalized map coordinates for our screen
var xS = function (var rx);
var yS = function (var ry);
// Converts a raw value into a normalized heatmap color
var color = function (var value);
// Use the d3 CSV viewer on our heatmap dataset, load in data
d3.csv("data/HM_"+tier+"_"+filter+".csv", function (error, data) {
	// Convert every value (loaded as string) into integers
	data.forEach(function(d){
	    d.x = parseInt(d.x);
	    d.y = parseInt(d.y);
	    d.v = parseInt(d.v);
	});
	// For every point, add a rectangle to our SVG element
	dataset = container.selectAll("rect")
	    .data(data)
	    .enter()
	    .append("rect")
	    .attr("x", function (d) { return xS(d.x)})
	    .attr("y", function (d) { return yS(d.y)})
	    .attr("width",   6 ) 
	    .attr("height",  6 )
	    .style("fill",   function (d) { return color(d.v) });
});
\end{lstlisting}