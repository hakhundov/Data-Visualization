\section{Heatmap}
One of the most easily created and useful spatial data visualizations is a \emph{heatmap}. In our DotA2 dataset we have several matches worth of player coordinates over time. Even more interesting, we have matches from several tier brackets. Once we have a heatmap of all data, it would also be interesting to create a heatmap per tier, to compare the overall map usage per tier. After that several filters were applied to these heatmaps.

When we tried to parse the dataset in the D3 library we encountered a severe problem however: this library is not suited for datasets of this scale. To solve this issue we chose to preprocess our dataset, transforming it into a heatmap dataset usable in D3 using a different platform first. Our second step was to filter only a certain tier in this transformation. The third step was applying several visual filters to these heatmaps.

\subsection{Preprocessing [TODO]}
TODO

\subsection{Filtering [TODO]}
TODO

\subsection{Rendering the heatmap in D3}
Since our heatmap dataset consisted of a csv with for every point a triple of the following format: $(x, y, v)$ where x and y encoded the location of this point, and v the count of this position being entered in the dataset by a player, loading the heatmap in D3 was trivial.

\begin{lstlisting}[language=JavaScript]
// Maps raw map coordinates into normalized map coordinates for our screen
var xS = function (var rx);
var yS = function (var ry);
// Converts a raw value into a normalized heatmap color
var color = function (var value);
// Use the d3 CSV viewer on our heatmap dataset, load in data
d3.csv("data/HM_"+tier+"_"+filter+".csv", function (error, data) {
	// Convert every value (loaded as string) into integers
	data.forEach(function(d){
	    d.x = parseInt(d.x);
	    d.y = parseInt(d.y);
	    d.v = parseInt(d.v);
	});
	// For every point, add a rectangle to our SVG element
	dataset = container.selectAll("rect")
	    .data(data)
	    .enter()
	    .append("rect")
	    .attr("x", function (d) { return xS(d.x)})
	    .attr("y", function (d) { return yS(d.y)})
	    .attr("width",   6 ) 
	    .attr("height",  6 )
	    .style("fill",   function (d) { return color(d.v) });
});
\end{lstlisting}